\ifdefined\COMPILINGMAIN
% Main file is compiling this section, skip the preamble
\else
% Individual file compilation
\documentclass[11pt]{article}
% Geometry and page layout
\usepackage{geometry}
\geometry{verbose,tmargin=3.375cm,bmargin=2cm,lmargin=3.375cm,rmargin=3.375cm}

% Input encoding and font settings
\usepackage[utf8]{inputenc}

% other fonts
%Slightly more bold
% \usepackage{mlmodern}
% \usepackage[T1]{fontenc}

%Moder modern look
% \usepackage{libertine}
% \usepackage{libertinust1math}
% \usepackage[T1]{fontenc}

\usepackage{amsfonts, amsmath, amsthm, bbm, setspace}
\onehalfspacing
\usepackage{algorithm2e}
\usepackage{tcolorbox} % For the grey background
% Create a tcolorbox style for the algorithm
\tcbuselibrary{listingsutf8}
\tcbset{
    algobox/.style={
        colback=gray!3, % Background color
        colframe=black,  % Border color
        sharp corners,   % Square corners
        boxrule=0.5pt,   % Border thickness
        before skip=10pt, % Vertical spacing before box
        after skip=10pt,  % Vertical spacing after box
        width=\textwidth, % Box width
    }
}

% Adjust algorithm2e settings for a similar look
\SetKwInOut{Input}{Input}
\SetKwInOut{Result}{Result}
\SetKwFor{For}{for}{:}{end}

% Adjust settings for algorithm2e
\SetAlgoCaptionSeparator{.} % Separator for caption
\SetAlgoNlRelativeSize{-2}  % Adjust line number font size
\SetAlgoInsideSkip{2pt}    % Reduce space between lines
\SetAlCapSkip{0pt}         % Remove extra space after the caption
% Ensure captions are above algorithms
\SetAlgoCaptionLayout{center} % Center caption
% Adjust the style of the algorithm to remove bottom line
\RestyleAlgo{ruled}
\SetAlCapSkip{0.5em}       % Space after caption
\SetAlgoVlined              % Ensures no horizontal lines at the end

% Theorem and math environments
\newtheorem{assumption}{Assumption}
\newtheorem{lemma}{Lemma}
\newtheorem{theorem}{Theorem}

% New math commands
\newcommand{\npsym}{\mathrel{\ooalign{\raisebox{.6ex}{$>$}\cr\raisebox{-.6ex}{$<$}}}}

% Table formatting
\usepackage{booktabs, multirow, array, tabularx}
\newcolumntype{N}{>{\centering\arraybackslash}m{.85in}}

% Caption settings
\usepackage{caption}
\captionsetup{format=plain, font=footnotesize, labelfont=bf,width=3.5in}
\setlength{\abovecaptionskip}{3pt plus 3pt minus 3pt}

% Figures and floats setup
\usepackage{graphicx, adjustbox,subcaption}
\usepackage{floatrow}
\floatsetup[figure]{capposition=top}
\floatsetup[table]{capposition=top}
\renewcommand\thefigure{\thesection.\arabic{figure}}
% Path to figures
\graphicspath{{../Figures/}}
\usepackage{tikz} % TikZ for creating figures
% URLs and references and colors
\usepackage[dvipsnames]{xcolor}
\usepackage[hyphens]{url}
\usepackage{hyperref}
\hypersetup{
    colorlinks=true,
    citecolor=[HTML]{901A1E}, %KU red
    linkcolor=[HTML]{901A1E}, %KU red    
    filecolor=blue, 
    urlcolor=[HTML]{901A1E}, %KU red
    hyperindex=true,
    hyperfigures=true,
    hyperfootnotes=true,
}

% Biblatex settings for references
\usepackage[style=authoryear, dashed=false, backend=bibtex]{biblatex}
\addbibresource{../Ref.bib}

\renewbibmacro*{volume+number+eid}{%
  \printfield{volume}%
  \setunit*{\addcomma\space}%
  \printfield{number}%
  \setunit{\addcomma\space}%
  \printfield{eid}
}
\DeclareFieldFormat[article]{volume}{\bibstring{volume}~#1}
\DeclareFieldFormat[article]{number}{\bibstring{number}~#1}
\DefineBibliographyStrings{english}{volume = {Vol.}, number = {No.}}

% Author name formatting
\DeclareNameAlias{author}{last-first}
\renewcommand*{\finalnamedelim}{\addspace and\space}
\renewcommand*{\multinamedelim}{\addcomma\space}

% Footnotes and appendix setup
\usepackage[hang,flushmargin]{footmisc}
\usepackage[toc,page]{appendix}
\renewcommand\appendixtocname{Appendices}
\renewcommand\appendixpagename{Appendices}

%# Assumptions like theorems and corrolaries
% {
%   \theoremstyle{plain}
%   \newtheorem{assumption}{Assumption}
% }
% Title setup
\usepackage{titlepic}
\usepackage{titlesec}
\titleformat{\section}{\normalfont\Large\bfseries}{\thesection}{1em}{}[{\titlerule[0.1pt]}]
% no text above figures!!!!
\usepackage{placeins}

% Abbreviations (acronym package)
\usepackage{acro}
\acsetup{list/name = Abbreviations}
\DeclareAcronym{PML}{short=PML, long= Probabilistic Machine Learning}
\DeclareAcronym{NTR}{short=NTR, long=No-Trade Region}
\DeclareAcronym{MC}{short=MC, long=Monte Carlo}
\DeclareAcronym{QMC}{short=QMC, long=Quasi-Monte Carlo}
\DeclareAcronym{RQMC}{short=RQMC, long=Randomized Quasi-Monte Carlo}
\DeclareAcronym{LDS}{short = LDS, long = Low-Discrepancy Sequences}
\DeclareAcronym{LLN}{short = LLN, long = Law of Large Numbers}
\DeclareAcronym{GPR}{short = GPR, long = Gaussian process regression}
\DeclareAcronym{GP}{short = GP, long = Gaussian process}
\DeclareAcronym{ARD}{short = ARD, long = Automatic Relevance Detection}
\DeclareAcronym{LOVE}{short = LOVE, long = LanczOS Variance Estimates}
\DeclareAcronym{SKIP}{short = SKIP, long = Structured Kernel Interpolation for Products}
\DeclareAcronym{SGD}{short = SGD, long = Stochastic Gradient Descent}
\DeclareAcronym{DP}{short = DP, long = Dynamic Programming}
\DeclareAcronym{MPT}{short=MPT, long=Modern Portfolio Theory}


% Conditional macro for compiling individual files
\ifdefined\COMPILINGMAIN
% Define settings when compiling the main document
\else
% Define minimal preamble for individual file compilation
\usepackage{geometry}
\geometry{verbose,tmargin=3.375cm,bmargin=2cm,lmargin=3.375cm,rmargin=3.375cm}
\fi

\AtBeginDocument{%
    \renewcommand{\contentsname}{Table of Contents}
    \renewcommand{\abstractname}{Abstract}
}
\setlength\parindent{11pt}
% Define the macro for compiling the main file
%\def\COMPILINGMAIN{}  % Include the main preamble
\begin{document}
\fi

\section{Introduction}\label{sec:Introduction}

Dynamic portfolio choice problems consider the optimal portfolio construction over time.
These have a general solution in the absence of market frictions.
When frictions are introduced, the problem becomes significantly more realistic, as 
investors face costs when trading assets.
However, this increased realism comes at a tradeoff of increased complexity in the problem,
as the optimal portfolio construction is no longer trivial to find.
In Dynamic Portfolio choice \ac{DP} schemes have been implemented to solve these problems numerically,
but the computational complexity of these schemes suffer from the curse of dimensionality
in a multitude of ways using multiple grid-based methods. In this regard the work of \textcite{Scheidegger2023} is of particular interest,
as they develop a computational framework which reduces the need for grid-based methods.
While much work has been put to developing a computational framework which reduces the need for grid-based methods,
this has not been applied to a broader set of portfolio choice models, 
and we therefore only have a limited idea of the scope of applicability of these methods.

I therefore extend the framework of \textcite{Scheidegger2023}, to new asset types and new cost functions,
to broaden the scope of models which can be solved using this framework, and to provide a broader understanding of the class of dynamic portfolio choice problems.
I analyse the impact of introducing various transaction costs types, such as fixed costs, and asset specific costs,
including the propotional transaction costs often seen in the litterature.
Furthermore i broaden the investment universe to include multiple asset types, such as stocks, bonds and vanilla options.
This paper therefore aims to provide a broader understanding of the class of dynamic portfolio choice problems,
utilizing the newest insights in computational methods seen in the litterature.

Furthermore a novel extension to the computational framework is provided, which aims to
reduce the computational burden in higher dimensions.
The framework suffers in higher dimensions, as the number of grid points increase, but also because the function approximation which leverages \ac{GP} becomes more complex.
I introduce \ac{SKIP}, which has been shown to increase the efficiency of the \ac{GP} when dimensionality is increased.

% I implement the computational framework of \textcite{Scheidegger2023}, which leverages
% \ac{GP} regression to approximate the value function of the dynamic portfolio choice problem,
% which has otherwise been prone to the curse of dimensionality in past implementations. 
% I likewise leverage their sampling scheme and quadrature methods to reduce the computational complexity of the problem.

I implement this framework on parametization analyzed earlier in the litterature, and compare the results to the existing literature.
Following this i extend the framework to include options, as seeen in \textcite{CaiJuddXu2020},
and new cost functions, as seen in \textcite{Dybvig2020}.


\ifdefined\COMPILINGMAIN
% Main file is compiling this section, skip the end
\else
\printbibliography
\end{document}
\fi