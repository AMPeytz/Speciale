\ifdefined\COMPILINGMAIN
% Main file is compiling this section, skip the preamble
\else
% Individual file compilation
\documentclass[11pt]{article}
% Geometry and page layout
\usepackage{geometry}
\geometry{verbose,tmargin=3.375cm,bmargin=2cm,lmargin=3.375cm,rmargin=3.375cm}

% Input encoding and font settings
\usepackage[utf8]{inputenc}
\usepackage{amsfonts, amsmath, amsthm, bbm, setspace}
\onehalfspacing

% Theorem and math environments
\newtheorem{assumption}{Assumption}
\newtheorem{lemma}{Lemma}
\newtheorem{theorem}{Theorem}

% New math commands
\newcommand{\npsym}{\mathrel{\ooalign{\raisebox{.6ex}{$>$}\cr\raisebox{-.6ex}{$<$}}}}

% Table formatting
\usepackage{booktabs, multirow, array, tabularx}
\newcolumntype{N}{>{\centering\arraybackslash}m{.85in}}

% Caption settings
\usepackage{caption}
\captionsetup{format=plain, font=footnotesize, labelfont=bf,width=3.5in}
\setlength{\abovecaptionskip}{3pt plus 3pt minus 3pt}

% Figures and floats setup
\usepackage{graphicx, adjustbox}
\usepackage{floatrow}
\floatsetup[figure]{capposition=top}
\floatsetup[table]{capposition=top}
\renewcommand\thefigure{\thesection.\arabic{figure}}
% Path to figures
\graphicspath{{../Figures/}}

% URLs and references and colors
\usepackage[dvipsnames]{xcolor}
\usepackage[hyphens]{url}
\usepackage{hyperref}
\hypersetup{
    colorlinks=true,
    citecolor=[HTML]{901A1E}, %KU red
    linkcolor=[HTML]{901A1E}, %KU red    
    filecolor=blue, 
    urlcolor=[HTML]{901A1E}, %KU red
    hyperindex=true,
    hyperfigures=true,
    hyperfootnotes=true,
}

% Biblatex settings for references
\usepackage[style=authoryear, dashed=false, backend=bibtex]{biblatex}
\addbibresource{../Ref.bib}

\renewbibmacro*{volume+number+eid}{%
  \printfield{volume}%
  \setunit*{\addcomma\space}%
  \printfield{number}%
  \setunit{\addcomma\space}%
  \printfield{eid}
}
\DeclareFieldFormat[article]{volume}{\bibstring{volume}~#1}
\DeclareFieldFormat[article]{number}{\bibstring{number}~#1}
\DefineBibliographyStrings{english}{volume = {Vol.}, number = {No.}}

% Author name formatting
\DeclareNameAlias{author}{last-first}
\renewcommand*{\finalnamedelim}{\addspace and\space}
\renewcommand*{\multinamedelim}{\addcomma\space}

% Footnotes and appendix setup
\usepackage[hang,flushmargin]{footmisc}
\usepackage[toc,page]{appendix}
\renewcommand\appendixtocname{Appendices A-F}
\renewcommand\appendixpagename{Appendices}

% Title setup
\usepackage{titlepic}
\usepackage{titlesec}
\titleformat{\section}{\normalfont\Large\bfseries}{\thesection}{1em}{}[{\titlerule[0.1pt]}]

% Abbreviations (acronym package)
\usepackage{acro}
\acsetup{list/name = Abbreviations}
\DeclareAcronym{MPT}{short=MPT, long=modern portfolio theory}
\DeclareAcronym{NTR}{short=NTR, long=no-trade-region}
\DeclareAcronym{MC}{short=MC, long=Monte Carlo}
\DeclareAcronym{QMC}{short=MPT, long=quasi-Monte Carlo}
\DeclareAcronym{RQMC}{short=MPT, long=randomized quasi-Monte Carlo}
\DeclareAcronym{LDS}{short = LDS, long = low-discrepancy sequences}
\DeclareAcronym{LLN}{short = LLN, long = law of large numbers}
\DeclareAcronym{GPR}{short = GPR, long = Gaussian process regression}
\DeclareAcronym{GP}{short = GP, long = Gaussian process}
\DeclareAcronym{ARD}{short = ARD, long = automatic relevance detection}
\DeclareAcronym{LOVE}{short = LOVE, long = LanczOS Variance estimates}
\DeclareAcronym{SKIP}{short = SKIP, long = Structured Kernel Interpolation for Products}
\DeclareAcronym{SGD}{short = SGD, long = stochastic gradient descent}
\DeclareAcronym{DP}{short = DP, long = dynamic programming}



% Conditional macro for compiling individual files
\ifdefined\COMPILINGMAIN
% Define settings when compiling the main document
\else
% Define minimal preamble for individual file compilation
\usepackage{geometry}
\geometry{verbose,tmargin=3.375cm,bmargin=2cm,lmargin=3.375cm,rmargin=3.375cm}
\fi

\AtBeginDocument{%
    \renewcommand{\contentsname}{Table of Contents}
    \renewcommand{\abstractname}{Abstract}
}
\setlength\parindent{11pt}
% Define the macro for compiling the main file
%\def\COMPILINGMAIN{}  % Include the main preamble
\begin{document}
\fi

\section{Introduction}\label{sec:Introduction}

Dynamic portfolio choice problems consider the optimal portfolio construction over time.
These have a general solution in the absence of market frictions.
When frictions are introduced, the problem is more realistic, as 
investors face costs when trading assets.
However, this increased realism comes at a tradeoff of increased complexity,
as the optimal portfolio construction is no longer trivial to find.
Investors typically face two types of costs when trading assets: fixed costs and proportional costs.

Proportional costs have been studied extensively in the literature, and optimal portfolio construction is well understood.
Fixed costs, on the other hand, have been less studied and the optimal portfolio construction is not as well defined,
especially so, when the asset returns are correlated or the number of risky assets is greater than two.

% This thesis aims to solve dynamic portfolio choice problems with fixed, proportional or both types of transaction costs, and correlated return structures.

For dynamic portfolio choice models with proportional costs, \ac{DP} schemes have been implemented to solve these problems numerically,
but the computational complexity of these schemes suffer from the curse of dimensionality
in a multitude of ways using multiple grid-based methods. In this regard, the work of \autocite{Scheidegger2023} is of particular interest to me,
as they develop a computational framework which reduces the need for grid-based methods.
While much work has been put into developing a computational framework which reduces the need for grid-based methods,
this has not been applied to a broader set of portfolio choice models, 
and the scope of applicability of these methods has not been fully explored.

I therefore extend the framework of \autocite{Scheidegger2023} to include fixed costs with correlated asset returns.
I do so to broaden the scope of models which can be solved and to provide clarity on this class of dynamic portfolio choice problems.

I display the impact of fixed costs on the optimal portfolio construction, and how this differs from proportional costs.
In doing so, I provide a novel approach to solving the fixed costs problem, based on the state of the art framework for proportional costs.
My approach leverages the geometric properties of the No-Trade Region (NTR), stemming from the trade frictions.
This alleviates the computational burden of the problem. However, I find that the fixed costs problem is more complex to solve than the proportional costs problem.

I find that the NTR under fixed costs schemes, is circular when asset returns are uncorrelated and elliptical when they are correlated.
This is in contrast to the NTR under proportional costs, which is a convex polytope.

I furthermore solve the fixed cost problem for three risky assets, which has not been done in the literature to my knowledge, and show that the resulting geometric shape of the NTR,
is the direct higher order generalization of the two risky assets case.

In addition to this I provide an intuitive approach, to solving new cost structures, not considered in this thesis, and how to adapt the computational approach to these.
This paves the way for future research in this area.

This thesis is structured as follows. In Section \ref{sec: literature}, I review the literature on dynamic portfolio choice problems with transaction costs,
so my contributions can be placed in context of the existing literature.
In Section \ref{Section: Economic-theory}, I cover the theoretical framework of the model, relevant assumptions and the economic intuition behind the model.
I also present the general class of problems which this thesis aims to solve.
In Section \ref{Section: Implmentation-details}, I present the computational framework, and how it is implemented for the proportional cost case.
In Section \ref{Section: Results}, I present the results of the model. I first cover the results of the proportional cost case and then the fixed cost case.
For the latter I present how the computational framework is extended to include fixed costs, and how the results differ from the proportional cost case.
I also combine the two costs and present the results of this.
In Section \ref{Section: Discussion}, I discuss the applicability of the model, the scalability of the proposed solution framework and future research avenues.
I lastly conclude on the findings of this thesis.

\ifdefined\COMPILINGMAIN
% Main file is compiling this section, skip the end
\else
\printbibliography
\end{document}
\fi