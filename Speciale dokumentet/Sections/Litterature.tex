\ifdefined\COMPILINGMAIN
% Main file is compiling this section, skip the preamble
\else
% Individual file compilation
\documentclass[11pt]{article}
% Geometry and page layout
\usepackage{geometry}
\geometry{verbose,tmargin=3.375cm,bmargin=2cm,lmargin=3.375cm,rmargin=3.375cm}

% Input encoding and font settings
\usepackage[utf8]{inputenc}

% other fonts
%Slightly more bold
% \usepackage{mlmodern}
% \usepackage[T1]{fontenc}

%Moder modern look
% \usepackage{libertine}
% \usepackage{libertinust1math}
% \usepackage[T1]{fontenc}

\usepackage{amsfonts, amsmath, amsthm, bbm, setspace}
\onehalfspacing
\usepackage{algorithm2e}
\usepackage{tcolorbox} % For the grey background
% Create a tcolorbox style for the algorithm
\tcbuselibrary{listingsutf8}
\tcbset{
    algobox/.style={
        colback=gray!3, % Background color
        colframe=black,  % Border color
        sharp corners,   % Square corners
        boxrule=0.5pt,   % Border thickness
        before skip=10pt, % Vertical spacing before box
        after skip=10pt,  % Vertical spacing after box
        width=\textwidth, % Box width
    }
}

% Adjust algorithm2e settings for a similar look
\SetKwInOut{Input}{Input}
\SetKwInOut{Result}{Result}
\SetKwFor{For}{for}{:}{end}

% Adjust settings for algorithm2e
\SetAlgoCaptionSeparator{.} % Separator for caption
\SetAlgoNlRelativeSize{-2}  % Adjust line number font size
\SetAlgoInsideSkip{2pt}    % Reduce space between lines
\SetAlCapSkip{0pt}         % Remove extra space after the caption
% Ensure captions are above algorithms
\SetAlgoCaptionLayout{center} % Center caption
% Adjust the style of the algorithm to remove bottom line
\RestyleAlgo{ruled}
\SetAlCapSkip{0.5em}       % Space after caption
\SetAlgoVlined              % Ensures no horizontal lines at the end

% Theorem and math environments
\newtheorem{assumption}{Assumption}
\newtheorem{lemma}{Lemma}
\newtheorem{theorem}{Theorem}

% New math commands
\newcommand{\npsym}{\mathrel{\ooalign{\raisebox{.6ex}{$>$}\cr\raisebox{-.6ex}{$<$}}}}

% Table formatting
\usepackage{booktabs, multirow, array, tabularx}
\newcolumntype{N}{>{\centering\arraybackslash}m{.85in}}

% Caption settings
\usepackage{caption}
\captionsetup{format=plain, font=footnotesize, labelfont=bf,width=3.5in}
\setlength{\abovecaptionskip}{3pt plus 3pt minus 3pt}

% Figures and floats setup
\usepackage{graphicx, adjustbox,subcaption}
\usepackage{floatrow}
\floatsetup[figure]{capposition=top}
\floatsetup[table]{capposition=top}
\renewcommand\thefigure{\thesection.\arabic{figure}}
% Path to figures
\graphicspath{{../Figures/}}
\usepackage{tikz} % TikZ for creating figures
% URLs and references and colors
\usepackage[dvipsnames]{xcolor}
\usepackage[hyphens]{url}
\usepackage{hyperref}
\hypersetup{
    colorlinks=true,
    citecolor=[HTML]{901A1E}, %KU red
    linkcolor=[HTML]{901A1E}, %KU red    
    filecolor=blue, 
    urlcolor=[HTML]{901A1E}, %KU red
    hyperindex=true,
    hyperfigures=true,
    hyperfootnotes=true,
}

% Biblatex settings for references
\usepackage[style=authoryear, dashed=false, backend=bibtex]{biblatex}
\addbibresource{../Ref.bib}

\renewbibmacro*{volume+number+eid}{%
  \printfield{volume}%
  \setunit*{\addcomma\space}%
  \printfield{number}%
  \setunit{\addcomma\space}%
  \printfield{eid}
}
\DeclareFieldFormat[article]{volume}{\bibstring{volume}~#1}
\DeclareFieldFormat[article]{number}{\bibstring{number}~#1}
\DefineBibliographyStrings{english}{volume = {Vol.}, number = {No.}}

% Author name formatting
\DeclareNameAlias{author}{last-first}
\renewcommand*{\finalnamedelim}{\addspace and\space}
\renewcommand*{\multinamedelim}{\addcomma\space}

% Footnotes and appendix setup
\usepackage[hang,flushmargin]{footmisc}
\usepackage[toc,page]{appendix}
\renewcommand\appendixtocname{Appendices}
\renewcommand\appendixpagename{Appendices}

%# Assumptions like theorems and corrolaries
% {
%   \theoremstyle{plain}
%   \newtheorem{assumption}{Assumption}
% }
% Title setup
\usepackage{titlepic}
\usepackage{titlesec}
\titleformat{\section}{\normalfont\Large\bfseries}{\thesection}{1em}{}[{\titlerule[0.1pt]}]
% no text above figures!!!!
\usepackage{placeins}

% Abbreviations (acronym package)
\usepackage{acro}
\acsetup{list/name = Abbreviations}
\DeclareAcronym{PML}{short=PML, long= Probabilistic Machine Learning}
\DeclareAcronym{NTR}{short=NTR, long=No-Trade Region}
\DeclareAcronym{MC}{short=MC, long=Monte Carlo}
\DeclareAcronym{QMC}{short=QMC, long=Quasi-Monte Carlo}
\DeclareAcronym{RQMC}{short=RQMC, long=Randomized Quasi-Monte Carlo}
\DeclareAcronym{LDS}{short = LDS, long = Low-Discrepancy Sequences}
\DeclareAcronym{LLN}{short = LLN, long = Law of Large Numbers}
\DeclareAcronym{GPR}{short = GPR, long = Gaussian process regression}
\DeclareAcronym{GP}{short = GP, long = Gaussian process}
\DeclareAcronym{ARD}{short = ARD, long = Automatic Relevance Detection}
\DeclareAcronym{LOVE}{short = LOVE, long = LanczOS Variance Estimates}
\DeclareAcronym{SKIP}{short = SKIP, long = Structured Kernel Interpolation for Products}
\DeclareAcronym{SGD}{short = SGD, long = Stochastic Gradient Descent}
\DeclareAcronym{DP}{short = DP, long = Dynamic Programming}
\DeclareAcronym{MPT}{short=MPT, long=Modern Portfolio Theory}


% Conditional macro for compiling individual files
\ifdefined\COMPILINGMAIN
% Define settings when compiling the main document
\else
% Define minimal preamble for individual file compilation
\usepackage{geometry}
\geometry{verbose,tmargin=3.375cm,bmargin=2cm,lmargin=3.375cm,rmargin=3.375cm}
\fi

\AtBeginDocument{%
    \renewcommand{\contentsname}{Table of Contents}
    \renewcommand{\abstractname}{Abstract}
}
\setlength\parindent{11pt}
% Define the macro for compiling the main file
%\def\COMPILINGMAIN{}  % Include the main preamble
\begin{document}
\fi


\section{Literature review}\label{sec: literature}
The purpose of this section is to review relevant literature to help understand the contributions made in this thesis, and their relation to the existing literature. 
This review covers dynamic portfolio choice problems, the introduction of transaction costs, and most notable contributions to the field.\\~\\
Modern theory on portfolio choice can be traced back to the mean-variance framework of Harry Markowitz, who
constructed and solved the static and single period, portfolio optimization problem, \autocite{Markowitz1952}. 

This covers the mean-variance framework which is the foundation of \ac{MPT}, suggesting investors should allocate wealth in order to maximize expected return, while minimizing exposure to risk.
Following this, the mean-variance framework has since been extended,
most notably by Robert Merton, who introduced a solution to the intertemporal portfolio choice problem in frictionless markets, \autocite{Merton1969}.
This solution is known as the Merton point in the asset allocation space or the Merton portfolio.
Merton's closed form solution suggests optimal asset allocations based on the asset return dynamics (mean-variance), and the risk aversion of the investor (preferences).
Later, an optimal consumption rule was found aswell \autocite{Merton1971}.

Multiple extensions have been made to the dynamic portfolio choice problem, such as the introduction of transaction costs,
adding realistic frictions to the problem, since trading assets incurs costs in the real world, and markets are not frictionless.
Most of the literature find that transaction costs create a region in the asset space, where it is sub-optimal to trade, known as the \ac{NTR}.

The literature on proportional costs in the dynamic portfolio choice problem is vast, whereas the fixed costs problem is less explored.
\autocite{morton1995optimal} analyse the problem with a fixed cost, relative to the investors wealth, and solve the problem numerically for two correlated risky assets.
They find a \ac{NTR} which is similar to an ellipse but with vertices. They conclude that the NTR is an ellipse.
\autocite{liu2002} solves the problem for uncorrelated assets with proportional and fixed costs and consumption. With fixed costs, No-Trade bounds are found for one risky asset in the shape of a conic.
Results differ from \autocite{morton1995optimal}, as the fixed cost NTR is not an ellipse but has corners. They conjecture this to be the case for correlated assets as well but skewed.
For proportional and fixed costs, multiple target portfolios are found inside the NTR, one for each corner, and the shape of the NTR is square.
\autocite{altarovici2015asymptotics} solve the dynamic problem for two uncorrelated risky assets with fixed costs.
They find that the \ac{NTR} is a slightly angled ellipsoid, using a differential equation approach.
\autocite{Dybvig2020} provide a comprehensive review of different transaction costs, and the implications of these on the optimal portfolio choice problem, however the setting is static.
They find that the \ac{NTR} from fixed costs with no correlation is circular, similar to the results of \autocite{morton1995optimal}.
From this, the excact shape of the fixed cost \ac{NTR} is not entirely clear. Most find an ellipsoid, but the skewness, connection to the correlation of the asset returns,
and whether the NTR has corners or not, is not entirely clear.
Furthermore, solutions in the literature are limited to two risky assets,
and the solution methods for the dynamic setting has not followed the same advances as the proportional costs problem.

\autocites{Zabel1973}{Constantinides1976}{Constantinides1986} find that for multiple preference types,
under proportional transaction costs, the investors decision depends on the the remaining life span, wealth and current allocation.
Transaction costs entail an \ac{NTR}, where the optimal reallocation decision for portfolios inside, is to do nothing, and for portfolios outside this region,
the optimal decision is to trade towards the boundary of the \ac{NTR}. This is a shift from Mertons framework, where constant trading toward the Merton allocation,
which is the optimal allocation in the absence of transaction costs, is optimal. Thus, transaction costs restrain
investors from acting optimally in the classical sense.
Numerical examples only cover the case of one risky asset with restrictions on the decision space and results remain qualitative or approximate.

Notably, \autocite{DavisNorman1990} derive explicit solutions for the case of a single risky asset.
They similarly find that proportional transaction costs lead to a \ac{NTR} around the Merton point and provide a solution algorithm for the stochastic control problem.

\autocite{Aikan1996} use a Bellman equation in the N-dimensional asset space, and provide further insight to the properties of the \ac{NTR}, however the problem is only solved for the case of two risky assets with one risk free asset.
Further analysis of this has been conducted extensively, e.g see \autocites{ShreveSoner1994}{Oksendal2002}{JanecekShreve2004}, however 
the asset space is still constrained or solutions remain asymptotic. 
\autocites{Muthuraman2006}{Muthuraman2008} tackle a three risky asset space and provide a numerical solution to the problem, using a finites differences method.

The paper by \autocite{CaiJuddXu2013}, which is central to this thesis, considers a more general setting with multiple risky assets and a risk-free asset
and provide a solution algorithm based on \ac{DP}, numerical integration and polynomial approximation.
They solve the dynamic problem for up to six risky assets,
and later introduce and solve the problem with novelties, such as stochastic asset parameters
or an option on a underlying asset in the portfolio \autocite{CaiJuddXu2020}. This work only considers proportional transaction costs and relies on a super computer to solve the problem.

The curse of dimensionality, which haunts the prior methods applied, is somewhat tackled by the use
of adaptive sparse grid methods and sparse quadratuture rules by \autocite{Schober2022}. However, results require the use of super computers. 
\autocite{Scheidegger2023} further reduce the computational burden by using a \ac{GPR},
to approximate value functions and provide a problem specific point-sampling strategy to reduce the number of points in the state space
needed to approximate the \ac{NTR}.
Increasing the dimensions of the asset space does still increase the dimensionality of the problem and the computational burden,
however this is at a much lower extent than previous methods. This is the most recent computational advance currently in the field, and is basis for the computational framework in this thesis.

\ifdefined\COMPILINGMAIN
% Main file is compiling this section, skip the end
\else
\printbibliography
\end{document}
\fi