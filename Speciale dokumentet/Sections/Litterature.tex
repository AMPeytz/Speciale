\ifdefined\COMPILINGMAIN
% Main file is compiling this section, skip the preamble
\else
% Individual file compilation
\documentclass[11pt]{article}
% Geometry and page layout
\usepackage{geometry}
\geometry{verbose,tmargin=3.375cm,bmargin=2cm,lmargin=3.375cm,rmargin=3.375cm}

% Input encoding and font settings
\usepackage[utf8]{inputenc}

% other fonts
%Slightly more bold
% \usepackage{mlmodern}
% \usepackage[T1]{fontenc}

%Moder modern look
% \usepackage{libertine}
% \usepackage{libertinust1math}
% \usepackage[T1]{fontenc}

\usepackage{amsfonts, amsmath, amsthm, bbm, setspace}
\onehalfspacing
\usepackage{algorithm2e}
\usepackage{tcolorbox} % For the grey background
% Create a tcolorbox style for the algorithm
\tcbuselibrary{listingsutf8}
\tcbset{
    algobox/.style={
        colback=gray!3, % Background color
        colframe=black,  % Border color
        sharp corners,   % Square corners
        boxrule=0.5pt,   % Border thickness
        before skip=10pt, % Vertical spacing before box
        after skip=10pt,  % Vertical spacing after box
        width=\textwidth, % Box width
    }
}

% Adjust algorithm2e settings for a similar look
\SetKwInOut{Input}{Input}
\SetKwInOut{Result}{Result}
\SetKwFor{For}{for}{:}{end}

% Adjust settings for algorithm2e
\SetAlgoCaptionSeparator{.} % Separator for caption
\SetAlgoNlRelativeSize{-2}  % Adjust line number font size
\SetAlgoInsideSkip{2pt}    % Reduce space between lines
\SetAlCapSkip{0pt}         % Remove extra space after the caption
% Ensure captions are above algorithms
\SetAlgoCaptionLayout{center} % Center caption
% Adjust the style of the algorithm to remove bottom line
\RestyleAlgo{ruled}
\SetAlCapSkip{0.5em}       % Space after caption
\SetAlgoVlined              % Ensures no horizontal lines at the end

% Theorem and math environments
\newtheorem{assumption}{Assumption}
\newtheorem{lemma}{Lemma}
\newtheorem{theorem}{Theorem}

% New math commands
\newcommand{\npsym}{\mathrel{\ooalign{\raisebox{.6ex}{$>$}\cr\raisebox{-.6ex}{$<$}}}}

% Table formatting
\usepackage{booktabs, multirow, array, tabularx}
\newcolumntype{N}{>{\centering\arraybackslash}m{.85in}}

% Caption settings
\usepackage{caption}
\captionsetup{format=plain, font=footnotesize, labelfont=bf,width=3.5in}
\setlength{\abovecaptionskip}{3pt plus 3pt minus 3pt}

% Figures and floats setup
\usepackage{graphicx, adjustbox,subcaption}
\usepackage{floatrow}
\floatsetup[figure]{capposition=top}
\floatsetup[table]{capposition=top}
\renewcommand\thefigure{\thesection.\arabic{figure}}
% Path to figures
\graphicspath{{../Figures/}}
\usepackage{tikz} % TikZ for creating figures
% URLs and references and colors
\usepackage[dvipsnames]{xcolor}
\usepackage[hyphens]{url}
\usepackage{hyperref}
\hypersetup{
    colorlinks=true,
    citecolor=[HTML]{901A1E}, %KU red
    linkcolor=[HTML]{901A1E}, %KU red    
    filecolor=blue, 
    urlcolor=[HTML]{901A1E}, %KU red
    hyperindex=true,
    hyperfigures=true,
    hyperfootnotes=true,
}

% Biblatex settings for references
\usepackage[style=authoryear, dashed=false, backend=bibtex]{biblatex}
\addbibresource{../Ref.bib}

\renewbibmacro*{volume+number+eid}{%
  \printfield{volume}%
  \setunit*{\addcomma\space}%
  \printfield{number}%
  \setunit{\addcomma\space}%
  \printfield{eid}
}
\DeclareFieldFormat[article]{volume}{\bibstring{volume}~#1}
\DeclareFieldFormat[article]{number}{\bibstring{number}~#1}
\DefineBibliographyStrings{english}{volume = {Vol.}, number = {No.}}

% Author name formatting
\DeclareNameAlias{author}{last-first}
\renewcommand*{\finalnamedelim}{\addspace and\space}
\renewcommand*{\multinamedelim}{\addcomma\space}

% Footnotes and appendix setup
\usepackage[hang,flushmargin]{footmisc}
\usepackage[toc,page]{appendix}
\renewcommand\appendixtocname{Appendices}
\renewcommand\appendixpagename{Appendices}

%# Assumptions like theorems and corrolaries
% {
%   \theoremstyle{plain}
%   \newtheorem{assumption}{Assumption}
% }
% Title setup
\usepackage{titlepic}
\usepackage{titlesec}
\titleformat{\section}{\normalfont\Large\bfseries}{\thesection}{1em}{}[{\titlerule[0.1pt]}]
% no text above figures!!!!
\usepackage{placeins}

% Abbreviations (acronym package)
\usepackage{acro}
\acsetup{list/name = Abbreviations}
\DeclareAcronym{PML}{short=PML, long= Probabilistic Machine Learning}
\DeclareAcronym{NTR}{short=NTR, long=No-Trade Region}
\DeclareAcronym{MC}{short=MC, long=Monte Carlo}
\DeclareAcronym{QMC}{short=QMC, long=Quasi-Monte Carlo}
\DeclareAcronym{RQMC}{short=RQMC, long=Randomized Quasi-Monte Carlo}
\DeclareAcronym{LDS}{short = LDS, long = Low-Discrepancy Sequences}
\DeclareAcronym{LLN}{short = LLN, long = Law of Large Numbers}
\DeclareAcronym{GPR}{short = GPR, long = Gaussian process regression}
\DeclareAcronym{GP}{short = GP, long = Gaussian process}
\DeclareAcronym{ARD}{short = ARD, long = Automatic Relevance Detection}
\DeclareAcronym{LOVE}{short = LOVE, long = LanczOS Variance Estimates}
\DeclareAcronym{SKIP}{short = SKIP, long = Structured Kernel Interpolation for Products}
\DeclareAcronym{SGD}{short = SGD, long = Stochastic Gradient Descent}
\DeclareAcronym{DP}{short = DP, long = Dynamic Programming}
\DeclareAcronym{MPT}{short=MPT, long=Modern Portfolio Theory}


% Conditional macro for compiling individual files
\ifdefined\COMPILINGMAIN
% Define settings when compiling the main document
\else
% Define minimal preamble for individual file compilation
\usepackage{geometry}
\geometry{verbose,tmargin=3.375cm,bmargin=2cm,lmargin=3.375cm,rmargin=3.375cm}
\fi

\AtBeginDocument{%
    \renewcommand{\contentsname}{Table of Contents}
    \renewcommand{\abstractname}{Abstract}
}
\setlength\parindent{11pt}
% Define the macro for compiling the main file
%\def\COMPILINGMAIN{}  % Include the main preamble
\begin{document}
\fi


\section{Literature review}\label{sec: literature}
The purpose of this section is to review relevant literature to help understand the contributions made in this thesis. 
This review covers \ac{MPT}, from its foundations and into the 21st century. \\
Modern theory on portfolio choice can be traced back to the mean-variance framework of Harry Markowitz, who
constructed and solved the now well established, static and single period, portfolio optimization problem, \textcite{Markowitz1952}. 
This covers the mean-variance framework which is the foundation of \ac{MPT}, suggesting investors should allocate wealth in order to maximize expected return, while minimizing exposure to risk.
Following this, the mean-variance framework has since been extended to a continious time setting,
most notably by Robert Merton, who introduced a solution to the intertemporal portfolio choice problem in frictionless markets, \textcite{Merton1969}, 
and later adding consumption rules aswell \textcite{Merton1971}. This solution is known as the Merton point in the asset allocation space, or the Merton portfolio.
Mertons closed form solution suggests optimal asset allocations based on the asset return dynamics (mean-variance), and the risk aversion of the investor (preferences). 
Hence in a continious time setting, the optimal allocation changes if the asset dynamics change.\\
Multiple extensions have been made to the classical dynamic portfolio choice problem, such as the introduction of transaction costs,
adding realistic constraints to the problem, since trading assets incurs costs in the real world, and markets are not frictionless.
\textcite{Zabel1973} adresses transaction costs with CRRA preferences, but is limited to a discrete time setting, a single risky asset and a small horizon.\\
\textcite{Constantinides1976,Constantinides1986} returns to the continious time setting, and find that for multiple preference types,
under proportional transaction costs. The investors decision then depends on the the remaining life span, wealth and current allocation.
Trading costs create a \ac{NTR}, where the optimal reallocation decision for portfolios inside this, is do to nothing, and for portfolios outside this region,
the optimal decision is to trade towards the boundary of the \ac{NTR}. This is a shift from Mertons framework, where constant trading toward the Merton allocation,
which is the optimal allocation in the absence of transaction costs, is optimal. Hence transaction costs restrain
investors from acting optimally in the classical sense.\\
Numerical examples only cover the case of one risky asset, with restrictions on the decision space, and results remain qualitative or approximate.
Notably \textcite{DavisNorman1990} derive explicit solutions for the case of a single risky asset.
They similarly find that proportional transaction costs lead to a \ac{NTR} around the Merton point, and provide a solution algorithm for the stochastic control problem.
This has later been made more rigorous such as 
\textcite{Aikan1996} who use a Hamilton-Jacobi-Bellman equation in the N-dimensional asset space, and provide further insight to the properties of the \ac{NTR}, however the problem is only solved for the case of $k=2$ risky assets with one risk free asset.
Further analysis of this has been conducted extensively, e.g see \textcite{ShreveSoner1994}, \textcite{Oksendal2002}, \textcite{JanecekShreve2004}, however 
the asset space is still constrained or solutions remain asymptotic. 
\textcite{Muthuraman2006,Muthuraman2008} tackle a $D=3$ risky asset space, and provide a numerical solution to the problem, using a finites differences.\\
The paper by \textcite{CaiJuddXu2013}, which is central to this thesis, consider a more general setting, with multiple risky assets and a risk-free asset,
and provides a solution algorithm, based on dynamic programming, numerical integration and polynomial approximation,
to solve the dynamic problem for up to $k=6$ risky assets and thus $D=7$ assets in total,
and later introduce and solve the problem with novelties, such as stochastic asset parameters
or an option on an underlying asset in the portfolio \textcite{CaiJuddXu2020}.
The curse of dimensionality, which haunts the prior methods applied, is somewhat tackled by the use
of adaptive sparse grid methods, and sparse quadratuture rules by \textcite{Schober2022}.\\
\textcite{Scheidegger2023} further reduces the computational burden by using a Gaussian process regression
to approximate value functions, and a problem specific point sampling strategy to reduce the number of points in the state space
needed to characterize the \ac{NTR}.
Increasing the dimensions of the asset space does still increase the dimensionality of the problem, and the computational burden,
however this is at a much lower extent than previous methods.\\
Beyond the analysis conducted by the authors above, several related avenues of reseatch have been conducted on the dynamic portfolio choice problemn.
\textcite{Garleanu2013} remains an influential paper, which aims to derive optimal closed form portfolio policy, when returns are driven by signals with mean reversion.
This provides an insightful analysis of how to trade towards the optimal portfolio, given quadratic transaction costs, within a set scope of serially correlated assets.
\textcite{Dybvig2020} provides a comprehensive overview on the usage of different transaction cost functions, hedging with futures and security specific costs.
Dybvig find that by changing the transaction cost function, the properties of the NTR is altered.\footnote{\textcite{Scheidegger2023}, also note that their framework is applicable to different transaction cost functions.}


\ifdefined\COMPILINGMAIN
% Main file is compiling this section, skip the end
\else
\printbibliography
\end{document}
\fi