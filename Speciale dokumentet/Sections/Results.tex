\ifdefined\COMPILINGMAIN
% Main file is compiling this section, skip the preamble
\else
% Individual file compilation
\documentclass[11pt]{article}
% Geometry and page layout
\usepackage{geometry}
\geometry{verbose,tmargin=3.375cm,bmargin=2cm,lmargin=3.375cm,rmargin=3.375cm}

% Input encoding and font settings
\usepackage[utf8]{inputenc}
\usepackage{amsfonts, amsmath, amsthm, bbm, setspace}
\onehalfspacing

% Theorem and math environments
\newtheorem{assumption}{Assumption}
\newtheorem{lemma}{Lemma}
\newtheorem{theorem}{Theorem}

% New math commands
\newcommand{\npsym}{\mathrel{\ooalign{\raisebox{.6ex}{$>$}\cr\raisebox{-.6ex}{$<$}}}}

% Table formatting
\usepackage{booktabs, multirow, array, tabularx}
\newcolumntype{N}{>{\centering\arraybackslash}m{.85in}}

% Caption settings
\usepackage{caption}
\captionsetup{format=plain, font=footnotesize, labelfont=bf,width=3.5in}
\setlength{\abovecaptionskip}{3pt plus 3pt minus 3pt}

% Figures and floats setup
\usepackage{graphicx, adjustbox}
\usepackage{floatrow}
\floatsetup[figure]{capposition=top}
\floatsetup[table]{capposition=top}
\renewcommand\thefigure{\thesection.\arabic{figure}}
% Path to figures
\graphicspath{{../Figures/}}

% URLs and references and colors
\usepackage[dvipsnames]{xcolor}
\usepackage[hyphens]{url}
\usepackage{hyperref}
\hypersetup{
    colorlinks=true,
    citecolor=[HTML]{901A1E}, %KU red
    linkcolor=[HTML]{901A1E}, %KU red    
    filecolor=blue, 
    urlcolor=[HTML]{901A1E}, %KU red
    hyperindex=true,
    hyperfigures=true,
    hyperfootnotes=true,
}

% Biblatex settings for references
\usepackage[style=authoryear, dashed=false, backend=bibtex]{biblatex}
\addbibresource{../Ref.bib}

\renewbibmacro*{volume+number+eid}{%
  \printfield{volume}%
  \setunit*{\addcomma\space}%
  \printfield{number}%
  \setunit{\addcomma\space}%
  \printfield{eid}
}
\DeclareFieldFormat[article]{volume}{\bibstring{volume}~#1}
\DeclareFieldFormat[article]{number}{\bibstring{number}~#1}
\DefineBibliographyStrings{english}{volume = {Vol.}, number = {No.}}

% Author name formatting
\DeclareNameAlias{author}{last-first}
\renewcommand*{\finalnamedelim}{\addspace and\space}
\renewcommand*{\multinamedelim}{\addcomma\space}

% Footnotes and appendix setup
\usepackage[hang,flushmargin]{footmisc}
\usepackage[toc,page]{appendix}
\renewcommand\appendixtocname{Appendices A-F}
\renewcommand\appendixpagename{Appendices}

% Title setup
\usepackage{titlepic}
\usepackage{titlesec}
\titleformat{\section}{\normalfont\Large\bfseries}{\thesection}{1em}{}[{\titlerule[0.1pt]}]

% Abbreviations (acronym package)
\usepackage{acro}
\acsetup{list/name = Abbreviations}
\DeclareAcronym{MPT}{short=MPT, long=modern portfolio theory}
\DeclareAcronym{NTR}{short=NTR, long=no-trade-region}
\DeclareAcronym{MC}{short=MC, long=Monte Carlo}
\DeclareAcronym{QMC}{short=MPT, long=quasi-Monte Carlo}
\DeclareAcronym{RQMC}{short=MPT, long=randomized quasi-Monte Carlo}
\DeclareAcronym{LDS}{short = LDS, long = low-discrepancy sequences}
\DeclareAcronym{LLN}{short = LLN, long = law of large numbers}
\DeclareAcronym{GPR}{short = GPR, long = Gaussian process regression}
\DeclareAcronym{GP}{short = GP, long = Gaussian process}
\DeclareAcronym{ARD}{short = ARD, long = automatic relevance detection}
\DeclareAcronym{LOVE}{short = LOVE, long = LanczOS Variance estimates}
\DeclareAcronym{SKIP}{short = SKIP, long = Structured Kernel Interpolation for Products}
\DeclareAcronym{SGD}{short = SGD, long = stochastic gradient descent}
\DeclareAcronym{DP}{short = DP, long = dynamic programming}



% Conditional macro for compiling individual files
\ifdefined\COMPILINGMAIN
% Define settings when compiling the main document
\else
% Define minimal preamble for individual file compilation
\usepackage{geometry}
\geometry{verbose,tmargin=3.375cm,bmargin=2cm,lmargin=3.375cm,rmargin=3.375cm}
\fi

\AtBeginDocument{%
    \renewcommand{\contentsname}{Table of Contents}
    \renewcommand{\abstractname}{Abstract}
}
\setlength\parindent{11pt}
% Define the macro for compiling the main file
%\def\COMPILINGMAIN{}  % Include the main preamble
\begin{document}
\fi

\section{Results} \label{Section: Results}
For the following results we consider 3 types of parameterizations for the portfolio problem.
The first is a simple case where the assets are identically distributed as seen in \autocite{CaiJuddXu2013},
the second is a case where the parameters are chosen to match the parameters in \autocite{Schober2022} also seen in \autocite{Scheidegger2023},
and the last is a modification of the first case where the correlation between the assets is larger (correlation coefficient of $0.75$).
\begin{table}[!ht]
    \label{table: Parameters_Base_Models}
    \centering
    \caption{Parameters for Examples of Portfolio Problems}
    \begin{tabular}{lccc}
    \toprule
    & \textbf{i.i.d Assets} & \textbf{Schober Parameters} & \textbf{High Correlation} \\
    \midrule
    $T$        & 6                & 6                & 6                \\
    $k$        & 3                & 5                & 3                \\
    $\gamma$   & 3.0              & 3.5              & 3.0              \\
    $\tau$     & 0.5\%            & 0.5\%            & 0.5\%            \\
    $\beta$    & 0.97             & 0.97             & 0.97             \\
    $r$        & $3$\% & $4$\%    &  $3$\% \\
    $\mu^\top$ & (0.07, 0.07) & $\mu_{\text{Shober}}$ & (0.07, 0.07) \\
    $\Sigma$   & 
    $\begin{bmatrix}
    0.04 & 0.00 \\
    0.00 & 0.04
    \end{bmatrix}$
    & $\Sigma_{\text{Schober}}$ 
    & 
    $\begin{bmatrix}
    0.04 & 0.03\\
    0.03 & 0.04
    \end{bmatrix}$ \\
    \bottomrule
    \end{tabular}
\end{table}
\[
\mu_{\text{Schober}}^\top = 
\begin{bmatrix}
0.0572 & 0.0638 & 0.07 & 0.0764 & 0.0828
\end{bmatrix}
\]
% Define the Schober covariance matrix
\[
\Sigma_{\text{Schober}} = 
\begin{bmatrix}
0.0256 & 0.00576 & 0.00288 & 0.00176 & 0.00096 \\
0.00576 & 0.0324 & 0.0090432 & 0.010692 & 0.01296 \\
0.00288 & 0.0090432 & 0.04 & 0.0132 & 0.0168 \\
0.00176 & 0.010692 & 0.0132 & 0.0484 & 0.02112 \\
0.00096 & 0.01296 & 0.0168 & 0.02112 & 0.0576 \\
\end{bmatrix}
\]

\subsection{Dynamic Portfolio Choice without consumption} \label{Subsection: NumericalIntegration}
I first consider the base model with proportional transaction costs
and no consumption. In the absence of consumption, the optimal portfolio is the merton points, which we plot in every figure.
I plot the No-trade region at time point 0 (initial time point) for each of the parameterizations in \autoref{fig:comparison_NTR}.
When using the Schober parameters we select the $d$ first elements of the mean vector, and truncate the covariance matrix to a $d \times d$ matrix,
depending on the number of assets $d$ in the model.
\begin{figure}[!ht]
    \centering
    % Top figure
    \begin{subfigure}[t]{\textwidth}
        \centering
        \includegraphics[scale=0.45]{NTR_Cai_Identical_d2_tau_0.005__no_consumption_t_0.png}
        \caption{No Trade Region for Independent Identically Distributed Assets.}
        \label{fig:NTR_2d_iid}
    \end{subfigure}

    % Space between the top and the bottom row
    \vspace{1em}

    % Bottom figures side by side
    \begin{subfigure}[t]{0.48\textwidth}
        \centering
        \includegraphics[scale=0.45]{NTR_Schober_Parameters_d2_tau_0.005__no_consumption_t_0.png}
        \caption{No Trade Region for Schober Parameters.}
        \label{fig:NTR_2d_Schober}
    \end{subfigure}%
    \hfill
    \begin{subfigure}[t]{0.48\textwidth}
        \centering
        \includegraphics[scale=0.45]{NTR_Cai_High_Correlation_d2_tau_0.005__no_consumption_t_0.png}
        \caption{No Trade Region for High Correlation.}
        \label{fig:NTR_2d_High_Correlation}
    \end{subfigure}

    % Overall caption
    \caption{Comparison of No Trade Regions.}
    \label{fig:comparison_NTR}
\end{figure}

\subsubsection{Investigating the No-Trade Region} \label{Subsubsection: InvestigatingNTR}
We now look at the No-Trade region for the base model with proportional transaction costs and no consumption in more detail.
Specifically we look at how the region behaves over the entire investment horizon $[0, T]$, and how the region changes with different transaction cost levels.
We choose to look at the model with the Schober parameters, as this is a mixture of the other two parameterizations.

\begin{figure}[!ht]
    \centering
    \includegraphics[scale = 0.45]{NTR_Schober_Parameters_d2_tau_0.005__no_consumptionOver_Time_t_5.png}
    \caption{No Trade Region for Schober Parameters over Time.}
    \label{fig:NTR_2d_iid_standalone}
    \floatfoot{The No-Trade region is plotted for the Schober parameters over the entire investment horizon $[0, T]$
    For time points $t\in [0,T-1]$ the \ac{NTR}s overlap.}
\end{figure}

\begin{figure}[!ht]
    \centering
    \includegraphics[scale = 0.45]{NTR_Cai_IdenticalDifferent_Tau_t_0.png}
    \caption{No Trade Region for the iid Parameters with different values of $\tau$.}
    \label{fig:NTR_2d_iid_tau_analysis}
\end{figure}

\subsubsection{Increasing the dimensionality of the model} \label{Subsubsection: IncreasingDimensionality}
We now increase the dimensionality of the model to $d = 3$ and look at the No-Trade region for the Schober parameters.




\ifdefined\COMPILINGMAIN
% Main file is compiling this section, skip the end
\else
% \printbibliography
\end{document}
\fi