\ifdefined\COMPILINGMAIN
% Main file is compiling this section, skip the preamble
\else
% Individual file compilation
\documentclass[11pt]{article}
% Geometry and page layout
\usepackage{geometry}
\geometry{verbose,tmargin=3.375cm,bmargin=2cm,lmargin=3.375cm,rmargin=3.375cm}

% Input encoding and font settings
\usepackage[utf8]{inputenc}

% other fonts
%Slightly more bold
% \usepackage{mlmodern}
% \usepackage[T1]{fontenc}

%Moder modern look
% \usepackage{libertine}
% \usepackage{libertinust1math}
% \usepackage[T1]{fontenc}

\usepackage{amsfonts, amsmath, amsthm, bbm, setspace}
\onehalfspacing
\usepackage{algorithm2e}
\usepackage{tcolorbox} % For the grey background
% Create a tcolorbox style for the algorithm
\tcbuselibrary{listingsutf8}
\tcbset{
    algobox/.style={
        colback=gray!3, % Background color
        colframe=black,  % Border color
        sharp corners,   % Square corners
        boxrule=0.5pt,   % Border thickness
        before skip=10pt, % Vertical spacing before box
        after skip=10pt,  % Vertical spacing after box
        width=\textwidth, % Box width
    }
}

% Adjust algorithm2e settings for a similar look
\SetKwInOut{Input}{Input}
\SetKwInOut{Result}{Result}
\SetKwFor{For}{for}{:}{end}

% Adjust settings for algorithm2e
\SetAlgoCaptionSeparator{.} % Separator for caption
\SetAlgoNlRelativeSize{-2}  % Adjust line number font size
\SetAlgoInsideSkip{2pt}    % Reduce space between lines
\SetAlCapSkip{0pt}         % Remove extra space after the caption
% Ensure captions are above algorithms
\SetAlgoCaptionLayout{center} % Center caption
% Adjust the style of the algorithm to remove bottom line
\RestyleAlgo{ruled}
\SetAlCapSkip{0.5em}       % Space after caption
\SetAlgoVlined              % Ensures no horizontal lines at the end

% Theorem and math environments
\newtheorem{assumption}{Assumption}
\newtheorem{lemma}{Lemma}
\newtheorem{theorem}{Theorem}

% New math commands
\newcommand{\npsym}{\mathrel{\ooalign{\raisebox{.6ex}{$>$}\cr\raisebox{-.6ex}{$<$}}}}

% Table formatting
\usepackage{booktabs, multirow, array, tabularx}
\newcolumntype{N}{>{\centering\arraybackslash}m{.85in}}

% Caption settings
\usepackage{caption}
\captionsetup{format=plain, font=footnotesize, labelfont=bf,width=3.5in}
\setlength{\abovecaptionskip}{3pt plus 3pt minus 3pt}

% Figures and floats setup
\usepackage{graphicx, adjustbox,subcaption}
\usepackage{floatrow}
\floatsetup[figure]{capposition=top}
\floatsetup[table]{capposition=top}
\renewcommand\thefigure{\thesection.\arabic{figure}}
% Path to figures
\graphicspath{{../Figures/}}
\usepackage{tikz} % TikZ for creating figures
% URLs and references and colors
\usepackage[dvipsnames]{xcolor}
\usepackage[hyphens]{url}
\usepackage{hyperref}
\hypersetup{
    colorlinks=true,
    citecolor=[HTML]{901A1E}, %KU red
    linkcolor=[HTML]{901A1E}, %KU red    
    filecolor=blue, 
    urlcolor=[HTML]{901A1E}, %KU red
    hyperindex=true,
    hyperfigures=true,
    hyperfootnotes=true,
}

% Biblatex settings for references
\usepackage[style=authoryear, dashed=false, backend=bibtex]{biblatex}
\addbibresource{../Ref.bib}

\renewbibmacro*{volume+number+eid}{%
  \printfield{volume}%
  \setunit*{\addcomma\space}%
  \printfield{number}%
  \setunit{\addcomma\space}%
  \printfield{eid}
}
\DeclareFieldFormat[article]{volume}{\bibstring{volume}~#1}
\DeclareFieldFormat[article]{number}{\bibstring{number}~#1}
\DefineBibliographyStrings{english}{volume = {Vol.}, number = {No.}}

% Author name formatting
\DeclareNameAlias{author}{last-first}
\renewcommand*{\finalnamedelim}{\addspace and\space}
\renewcommand*{\multinamedelim}{\addcomma\space}

% Footnotes and appendix setup
\usepackage[hang,flushmargin]{footmisc}
\usepackage[toc,page]{appendix}
\renewcommand\appendixtocname{Appendices}
\renewcommand\appendixpagename{Appendices}

%# Assumptions like theorems and corrolaries
% {
%   \theoremstyle{plain}
%   \newtheorem{assumption}{Assumption}
% }
% Title setup
\usepackage{titlepic}
\usepackage{titlesec}
\titleformat{\section}{\normalfont\Large\bfseries}{\thesection}{1em}{}[{\titlerule[0.1pt]}]
% no text above figures!!!!
\usepackage{placeins}

% Abbreviations (acronym package)
\usepackage{acro}
\acsetup{list/name = Abbreviations}
\DeclareAcronym{PML}{short=PML, long= Probabilistic Machine Learning}
\DeclareAcronym{NTR}{short=NTR, long=No-Trade Region}
\DeclareAcronym{MC}{short=MC, long=Monte Carlo}
\DeclareAcronym{QMC}{short=QMC, long=Quasi-Monte Carlo}
\DeclareAcronym{RQMC}{short=RQMC, long=Randomized Quasi-Monte Carlo}
\DeclareAcronym{LDS}{short = LDS, long = Low-Discrepancy Sequences}
\DeclareAcronym{LLN}{short = LLN, long = Law of Large Numbers}
\DeclareAcronym{GPR}{short = GPR, long = Gaussian process regression}
\DeclareAcronym{GP}{short = GP, long = Gaussian process}
\DeclareAcronym{ARD}{short = ARD, long = Automatic Relevance Detection}
\DeclareAcronym{LOVE}{short = LOVE, long = LanczOS Variance Estimates}
\DeclareAcronym{SKIP}{short = SKIP, long = Structured Kernel Interpolation for Products}
\DeclareAcronym{SGD}{short = SGD, long = Stochastic Gradient Descent}
\DeclareAcronym{DP}{short = DP, long = Dynamic Programming}
\DeclareAcronym{MPT}{short=MPT, long=Modern Portfolio Theory}


% Conditional macro for compiling individual files
\ifdefined\COMPILINGMAIN
% Define settings when compiling the main document
\else
% Define minimal preamble for individual file compilation
\usepackage{geometry}
\geometry{verbose,tmargin=3.375cm,bmargin=2cm,lmargin=3.375cm,rmargin=3.375cm}
\fi

\AtBeginDocument{%
    \renewcommand{\contentsname}{Table of Contents}
    \renewcommand{\abstractname}{Abstract}
}
\setlength\parindent{11pt}
% Define the macro for compiling the main file
%\def\COMPILINGMAIN{}  % Include the main preamble
\begin{document}
\fi
\section{Discussion} \label{Section: Discussion}
The dynamic portfolio choice problem with fixed or proportional costs has been introduced and solved in the previous sections.
The framework developed in this thesis, is able to solve the problem has been introduced in detail as well.
I will now discuss the results, and the applicability of the model, the scalability of the model, competing implementation methods, and avenues of future research,
in order to highlight the limitations of the model, and suggest areas for further research.

\subsection{Applicability of the model} \label{Subsection: Applicability of the model}
The model developed in this thesis is applicable to the problem of dynamic portfolio choice, with consumption, when investors face proportional transaction costs.
While the model is only able to solve a reduced investment universe, in the number of assets, the
results give a greater understanding on how transaction costs affect the optimal portfolio choice.
Investment towards the optimal allocation may in fact be more costly than the transaction costs themselves, and the model provides a framework for understanding this trade-off.
This is well known, and been studied in the litterature of computational finance and behavioural finance.

However, the since the results are only applicable to a reduced investment universe, the results may not be directly applicable to real world applications
and remain theoretical in nature. First of all the the model predicts optimal behavior given the distribution of the asset returns,
which are only available for past returns, and the distribution of future returns may differ from the distribution of past returns.
Hence, the model suffers from the same limitations as other models in the field of finance, in that it is based on historical data,
in my case simulated data even, and may not be applicable to future data.

Furthermore, even for parameters with high costs, as seen in prior sections, the simulation results showcase that only a miniscule amount of the wealth is lost due to fixed transaction costs.

For investors with a high wealth level, a pure fixed cost would be miniscule compared to the wealth level,
and the resulting \ac{NTR} would probably only be constructed from the proportional costs of which the implications were covered in section \ref{Subsection: Fixed_Proportional}.

Fixed cost \ac{NTR}s are therefore mostly relevant for investors with a low wealth level, or investors who rebalance frequently, maybe multiple times a day.
For these investors, the fixed cost is either relatively large, or incurred often.

For large institutional investors, the proportional cost \ac{NTR} can still be relevant, as the proportional costs scale with traded amount.

Until models with a large amount of assets, based on realistic distributional parameters\footnote{Which could stem from regressions on prior returns.} are available, the model will mostly be relevant in a theoretical setting,
or to give a general understanding of how transaction costs affect the optimal portfolio choice.

\subsection{Scalability of the model} \label{Subsection: Scalability of the model}
This thesis implements the framework constructed by \autocite{Scheidegger2023}
in order to solve the problem of dynamic portfolio choice, with consumption,
when investors face proportional transaction costs. This framework increases the scalability of the model to higher dimensions
than previously possible without the use of super computers, by minimizing the number of grid points needed to approximate the NTR.
However, the model is still computationally demanding, especially when the number of assets is high, and the number of grid points
needed to train the function approximators still increase with the dimensionality of the model.

Furthermore, the evaluation of the increasingly complex \ac{GP} increases exponentially in complexity with the number of assets.
Also, even for the most simple shape of the \ac{NTR} such as square, the number of vertices needed to effectively 
formulate the \ac{NTR} increases exponentially with the number of assets.
The framework is therefore not scalable to an arbitrary number of assets, and the number of assets that can be included in the model is limited by the computational power available.
This limits the use of the model in real world applications, where the number of assets considered is high.
Furthermore, distributional parameters would need to be estimated for each asset, which would further increase the computational complexity of
a real world application.

For fixed costs a novel algorithm, based on the work of \autocite{Scheidegger2023}, and the geometric properties
of the resulting \ac{NTR}s is developed. The algorithm is able to solve the problem, but introduces a new set of challenges.
While fewer initial points are needed, since I only need to approximate the center of the \ac{NTR} in the first case,
the bisection algorithm, needed to find find the edge of the \ac{NTR}, is computationally demanding, and re-introduces the the need
for evaluation at a fine grid along the trajectory from the center. Thus the fixed cost model poses further dimensional burden to the model.
Furthermore, the edge points needed to approximate the \ac{NTR} scales with dimensionality,
when assets are correlated, which in combination with the bisection algorithm introduces curse of dimensionality to the model.

Overall, a solution which can scale to a sufficiently large dimensionality, and which can be used in real world applications, is still yet to be found.
However, this paper does provide a step in the right direction, by providing a framework which can be used to solve the problem in a higher dimensionality than previously possible,
for the fixed cost case, which had previously not been solved for dimensions larger than $2$, in a dynamic setting.
\subsection{Competing implementation methods} \label{Subsection: Competing solution methods}
As noted in the prior section. The frameworks used and developev in this paper, face scalability issues.
Specifically the bisection algorithm used to find the edge of the \ac{NTR} in the fixed cost case, and the evaluation of the \ac{GP} in the proportional cost case.

For the function approximators, competing methods such as neural networks, and other machine learning methods, could be used to approximate the \ac{NTR}.
Neural networks are universal function approximators, and could potentially approximate the \ac{NTR} more efficiently than the \ac{GP}.
Since the goal is to approximate the \ac{NTR}, if a neural network could be implemented to approximate the \ac{NTR} more efficiently than the \ac{GP},
by skipping the evaluation of grid points, necessary for the \ac{GP}, the model could scale better.

The bisection algorithm used to find the edge of the \ac{NTR} in the fixed cost case, could likewise potentially be replaced by a more efficient algorithm.
The bisection algorithm is favoured in this paper, for its simplicity, and the fact that it is guaranteed to find the edge of the \ac{NTR}.
The algorithm is easy to understand in an intuitive manner, especially when the \ac{NTR} is presented geometrically.
The bisection algorithm could potentially be replaced by a root-finding algorithm leveraging the computed gradients in the current framework.
Such a solver should theoretically be able to find the edge of the \ac{NTR} more efficiently than the bisection algorithm, which needs to solve the model at each bisection mid point. 
\subsection{Avenues of Future Research} \label{Subsection: Future Research}
This thesis provides a framework for solving the problem of dynamic portfolio choice under various transaction costs and return structures.
By first, solving the problem over a fine grid, I find the geometric shape of the resulting \ac{NTR}, and then leverage this information to solve the problem more efficiently.
This framework can be extended to various types of transaction costs. Notably \autocite{Dybvig2020} considers asset specific fixed costs, and also price impact,
among other transaction costs not considered in this thesis. Future research could therefore consider other transaction cost structures, and combinations thereof,
and how these affect the optimal portfolio choice, by using the proposed framework.

Furthermore, the framework could be extended to consider other asset structures.
For example, \autocite{CaiJuddXu2020}, extend the model to include vanilla options on the assets considered in the model as well as butterfly options,
and \autocite{Dybvig2020} considers hedging with futures, albeit still in a static setting.
Futher research could consider how these asset structures affect the optimal portfolio choice, and how the framework can be extended to include these asset structures,
which if still computationally feasible, would be a novel contribution to the literature, as the case of futures as not been tackled in a dynamic setting,
and the case of options is computationally burdensome under the scheme of \autocite{CaiJuddXu2020}.
An analysis of price impact would likewise be interesting, specifically for large institutional investors,
whose trades can move the market, and thus affect the price of the assets they are trading. The impact on the \ac{NTR} in a dynamic setting would be interesting to see.

The consumption model could also be extended to include other types of goods, such as durable consumption goods, and how these affect the optimal portfolio choice.
One such type could be housing, which would borrow from the literature on household behaviour over the lifecycle, however I note that this would effectively move the focus of the thesis from finance household behaviour.


\section{Conclusion} \label{Section: Conclusion}
The purpose of this thesis was to solve dynamic portfolio choice problems with proportional and fixed transaction costs, and correlated return structures.
In this regard, I constructed a new solution method, based on the work of \autocite{Scheidegger2023}.

My framework solved the problem, with fixed costs relative to wealth, for more than two assets, which had not been done before to my knowledge, and provided a novel approach to the fixed cost problem.
The framework builds on the geometric properties of the \ac{NTR}, and combines the efficent solution method of \autocite{Scheidegger2023} to solve the problem.
I verified speculations on the geometric form of the fixed cost \ac{NTR}, and the implications of correlated return structures on this type \ac{NTR}.
The \ac{NTR} for fixed costs, with no correlations is indeed a sphere, and the \ac{NTR} for fixed costs with correlations is a ellipsoid, with angle and size dependent on the correlation structure.
I conducted a simulation study, which showed that a strategy which takes costs into account, can outperform a strategy which does not.

However, several limitations of the model remain, as discussed in Section \ref{Section: Discussion}.
The model’s applicability to real-world scenarios is constrained by its reliance on simulated data and reduced investment universes.
While it provides theoretical insights, the assumptions about return distributions and the computational demands limit its use for large-scale, real-world portfolios.
Additionally, the fixed-cost \ac{NTR} is most relevant for small investors or frequent traders, whereas proportional costs remain more applicable to other investors. 

I made three contributions along the way, which could prove useful for future research.
Firstly, I provided a novel approach to the fixed cost problem, based on the state of the art framework for proportional costs, which leverages the geometric shape of the \ac{NTR}.
Secondly, I solved dynamic portfolio choice problems with fixed costs for correlated assets, with more than two risky assets.
Thirdly, I presented an approach to solving new transaction cost structures, not yet considered, and how to adapt my computational approach to these.


\ifdefined\COMPILINGMAIN
% Main file is compiling this section, skip the end
\else
% \printbibliography
\end{document}
\fi